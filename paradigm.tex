\documentclass[11pt]{article}
\usepackage{amssymb}
\usepackage{amsmath,amscd,amsthm}
\usepackage[utf8]{inputenc}
\usepackage[english,russian]{babel}
\usepackage{pscyr}
\usepackage{indentfirst}
\usepackage{letltxmacro}
\usepackage{mathtools}
\usepackage{ulem}
\usepackage{xcolor}
%\thispagestyle{empty}
\usepackage{multirow}
\usepackage{amsmath}
\usepackage{comment}
%\usepackage{times}
\usepackage{tikz}
\usepackage[all]{xy}
\xyoption{tips}
\SelectTips{lu}{10}
 \usetikzlibrary{matrix,calc,arrows,shapes,snakes}

\def\be{\numberwithin{equation}{section}\begin{eqnarray}}
\def\ee{\end{eqnarray}}
\def\trademark{{\hbox{\tiny TM}}}
\def\dim{\textmd{dim} \hskip 3 pt}
\def\p{\partial}
\def\R{\Rightarrow}
\def\ph{\varphi}
\newtheorem{thm}{Theorem}[section]
\newtheorem{cor}[thm]{Corollary}
\newtheorem{lem}[thm]{Lemma}
\theoremstyle{remark}
\newtheorem{rem}[thm]{Remark}
\theoremstyle{definition}
\newtheorem{Def}[thm]{Definition}


\newcommand{\que}[1]{\footnote{\textcolor[rgb]{0.38,0.69,0.82}{#1}}}

\renewcommand\thefootnote{\textcolor[rgb]{0.38,0.69,0.82}{\arabic{footnote}}}


\LetLtxMacro{\oldsqrt}{\sqrt} % makes all sqrts closed
\renewcommand{\sqrt}[1][\ ]{%
  \def\DHLindex{#1}\mathpalette\DHLhksqrt}
\def\DHLhksqrt#1#2{%
  \setbox0=\hbox{$#1\oldsqrt[\DHLindex]{#2\,}$}\dimen0=\ht0
  \advance\dimen0-0.2\ht0
  \setbox2=\hbox{\vrule height\ht0 depth -\dimen0}%
  {\box0\lower0.71pt\box2}}

%$\sqrt[a]{b} \quad \oldsqrt[a]{b}$

%%%%%%%%%%%%%%%%%%%%%%%%%%%%%%%%%%%%%%%%%%%%%%%%%%%%%%%%%%%%%%%%%%%%%%%%
%%%%%%%%%               SPACE FILLING SETTINGS               %%%%%%%%%%%
%%%%%%%%%%%%%%%%%%%%%%%%%%%%%%%%%%%%%%%%%%%%%%%%%%%%%%%%%%%%%%%%%%%%%%%%
\textheight 23.5cm \textwidth 16.3cm
%\voffset=-1.2in
\voffset=-0.8in
%\voffset= - 1.85in
\hoffset= - 1.0in         % switch off for draft style
\multlinegap=1.0in
%%%%%%%%%%%%%%%%%%%%%%%%%%%%%%%%%%%%%%%%%%%%%%%%%%%%%%%%%%%%%%%%%%%%%%%%

\begin{document}

\baselineskip14pt
\bigskip


\tableofcontents
\bigskip
\bigskip

%======================================================

А.С. нам рассказывает вещи, которые не должны быть утеряны. Каждый раз он рассказывает, каковой (по-видимому) на самом деле должна быть парадигма. Здесь будет записываться <<передний фронт разработки>>, как в программировании. (Таким образом, это не конспект нашего семинара.)

Пожалуйста, \textbf{вносите изменения} в этот файл, \textbf{дополняйте} его. Пусть он станет вашим рабочим столом. Нет никакого другого способа что-то понять, чем попытаться изложить это листу бумаги/соседу/уточке в ванной. Но у изложения листу бумаги есть два дополнительных бонуса:
\begin{itemize}
  \item не пропадёт ваш скорбный труд и дум высокое стремленье
  \item возможность конструктивного фидбека от $n$ слушателей семинара
\end{itemize}

\subsubsection{Как вносить изменения в файл}

После некоторых размышлений я пришёл к выводу, что в качестве системы контроля версий мы будем использовать GitHub.

Если вы обнаруживаете, что чего-то не понимаете --- предлагаю задать вопрос сноской. Делается это с помощью синтаксиса $\backslash \text{que} \{ \text{почему?} \}$. Пример\que{почему?} заданного таким образом вопроса.

Если вы обнаруживаете, что что-то понимаете --- просто берёте и редактируете текст.

Если у вас не компилится этот tex-файл --- закомментируйте строку $\backslash\text{usepackage}\{ \text{pscyr} \}$ и попробуйте снова; пакет pscyr, содержащий красивые кириллические шрифты, не все себе устанавливали.

\newpage




\section{$A_{\infty}$-структуры, $\Delta_{BV}$-оператор, поливекторные поля}

\subsection{Наша задача --- понять, что вместо вопроса}

(Пока что наша задача --- великолепно переписать этот огрызок.)

Есть нечётные векторные поля $v_{\text{н}}$: $$v_{\text{н}} \in \oplus_k V^{\otimes k} \to V.$$
\que{Напомни, что такое нечётные векторные поля, то есть вставь нужное определение, что ли... Юра.}

И главное уравнение $A_{\infty}$-структуры имеет вид: $v_{\text{н}}^2 = 0.$ Или $\{v,v\}_G = 0.$

Но неправильно говорить просто об этом уравнении. Нужно ещё сфакторизовать по соотношению эквивалентности $v\sim v + \{v,w\}$ --- автоморфизмам этого векторного пространства.

$$v = v_0 + Pol, \hskip 20 pt Pol \ll v_0.$$ Получится уравнение $\{ v_0, Pol \} + \{ Pol, Pol \} = 0.$

Далее, можно сказать, что $Pol = P_0 + \omega$. Т.е. $v = v_0 + P_0 + \omega$. Именно $\omega$, этот чёртов третий член, даёт нам когомологии. Первые же два члена этого разложения определяют мир, который рассматривается.

Стоит отметить, что $v_0: V^{\otimes 2} \to V$ --- обычное умножение. Поэтому его обычно обозначают $m_2$.

Если мы работаем с кольцом $\mathbb{C}[x] \otimes \mathbb{C}[\lambda, \theta] / (\lambda \gamma \lambda)$ (пространственных переменных $x$ 10, чётных полей $\lambda$ 16, нечётных полей $\theta$ тоже 16), а в качестве $P_0$ мы берём $\lambda^{\alpha} \frac{\partial}{\partial \theta^{\alpha}}$, то получится $\mathcal{N} = 1$ $D=10$ SYM \que{интересно, какая из пяти?..}. Низкоэнергетическое приближение --- $D=10$ супергравитация.

Получается такой коммутативный квадрат:

%$$\xymatrix{ ? \ar@{.>}[r] \ar@{.>}[d] & \text{nearly comm.limit\que{что это?}} \ar[d] \\ \hbar\Delta_{BV} Pol+ \{Pol, Pol\} = 0 \ar[r] & \{Pol, Pol\} = 0}$$

\begin{center}
\begin{tikzpicture}[>=stealth',shorten >=1pt,auto,node distance=3cm,
  thick,back line/.style={densely dotted},
cross line/.style={preaction={draw=white, -,
line width=6pt}},main node/.style={circle,fill=blue!20,draw,font=\sffamily\Large\bfseries}]

\matrix (m) [matrix of math nodes,
row sep=3em, column sep=3em,
text height=1.5ex,
text depth=0.25ex]{
? &  & \text{nearly comm.limit} \\
 & & \\
\hbar\Delta_{BV} Pol+ \{Pol, Pol\} = 0 &  & \{Pol, Pol\} = 0 \\
};

% Сноску "что это?" из коммутативной диаграммы никак не поставишь --- я пытался. Поэтому так.

\path[->]
(m-1-1) edge [back line] (m-3-1)
(m-1-1) edge [back line] (m-1-3)
(m-3-1) edge [cross line] (m-3-3)
(m-1-3) edge [cross line] (m-3-3);
\end{tikzpicture}

\end{center}

И наша задача --- понять, что вместо вопроса.

\subsection{Задача матричной факторизации}

Точки на нашем пространстве модулей могут подъезжать к границе, и нужно предложить какое-то граничное условие. Граница является браной.

Сама по себе задача матричной факторизации давно известна в теории особенностей и сводится к элементарному утверждению. Именно, нужно решить матричное уравнение на $N$ вида $W = N^2$. $W(x)$ --- суперпотенциал, $N$ --- матрица, которую необходимо найти.\que{Тема матриц не раскрыта} Высший смысл происходящего формулируется так: триангулированные категории особенностей слоёв отображения $W$ эквивалентны категориям В-бран.


\begin{comment}
\footnotesize{}

Общеобразовательный кусок про деформации комплексных структур

$$Q_{\tau} = Q + \tau V_1 + \tau^2 V_2, \bar \p_{\tau} = \bar\p + \tau \mu_1 \p + \tau^2 \mu_2 \p + ...$$

Надо изучать комплексные структуры по модулю диффеоморфизмов. Оказывается, уравнение Кодаиры $$(\bar\p + \mu(\tau) \p)^2 =0$$ можно обобщить. А именно, можно деформировать не только дифференциал $\bar\p$, но и вообще всё что угодно.

В топологической квантовой механике имеем уравнение $$H = \{Q(\tau), G\}.$$

\normalsize{}

\end{comment}


\section{Новый взгляд на геометрию}
\subsection{Собственно взгляд}

Классическая геометрия является вырождением геометрии Полякова.

\begin{center}


\begin{comment}
\begin{tikzpicture}[
back line/.style={densely dotted},
cross line/.style={preaction={draw=white, -,
line width=6pt}}]
\end{comment}

\begin{tikzpicture}[>=stealth',shorten >=1pt,auto,node distance=3cm,
  thick,back line/.style={densely dotted},
cross line/.style={preaction={draw=white, -,
line width=6pt}},main node/.style={circle,fill=blue!20,draw,font=\sffamily\Large\bfseries}]


%\shade[shading=radial, inner color=blue]
%(0,0) rectangle (2,1);

 % \matrix[nodes={fill=blue!20},
%       row sep=7em,column sep=3em,text height=1.5ex,
% text depth=0.25ex] 
%{
%    \node[ellipse] {\text{яяя}}; \\};

\matrix (m) [matrix of math nodes,
row sep=3em, column sep=3em,
text height=1.5ex,
text depth=0.25ex]{
& \text{Геометрия} & & \text{Физика} \\
\text{гладкие мн-я} & & \text{классическая} \\
& \text{алг. геом.} & & \text{квант. мех.} \\
 & & & \textmd{некомм. геом.} \\
  & \text{Поляков} & &  \\
};

\draw (m-2-1) edge [back line] node {\tiny{\text{Риччи-плоск.;\hskip 20 pt инстантоны}}} (m-2-3)
(m-2-3) edge [back line] node {\tiny{\text{идеи Фейнмана}}} (m-3-4)
(m-3-2) edge [back line] node {\tiny{\text{некомм. алгебра}}} (m-3-4);
%(m-3-2) edge (m-4-1)
%(m-3-4) edge (m-4-3)
%(m-2-1) edge [back line] (m-4-1)

%(m-2-3) edge [back line] (m-4-3)
\path[->]
(m-1-4) edge (m-3-4)
(m-3-4) edge (m-4-4)
(m-1-2) edge [cross line] (m-3-2)
(m-2-3) edge [cross line] (m-5-2)
(m-4-4) edge (m-5-2)
(m-1-2) edge (m-2-1)
(m-1-4) edge (m-2-3);
%(m-4-1) edge (m-4-3)
\end{tikzpicture}

\end{center}



Есть две точки зрения на то, что изучает геометрия:

\begin{itemize}
  \item пространство
  \item пространство и какая-нибудь геометрическая структура на нём
\end{itemize}

Согласно Полякову, (гомотопическая) конформная теория --- это и есть пространство + геометрия на нём.

\subsection{Напоминание о том, что такое конформная теория поля}

Конформные теории поля --- теории поля, инвариантные относительно конформных преобразований метрики. В них есть <<основной объект>> $I$, зависящий от поверхности $\Sigma$ (вид поверхности зависит от того, открытая или замкнутая струна, а также от количества наблюдаемых $V_1$, ..., $V_n$ в рассматриваемом корреляторе). Метрик, сфакторизованных по соотношению эквивалентности <<конформно связанные метрики эквивалентны>>, столько же, сколько комплексных структур, поэтому основной объект $I$ зависит также от комплексной структуры (т.е. от дифференциала Бельтрами $\mu$, который параметризует модули комплексных структур на $\Sigma$). Комплексные структуры, очевидно, надо изучать по модулю диффеоморфизмов. Физический смысл нижеследующей картинки вот каков: для того, чтобы вычислить коррелятор операторов $V_1$, ..., $V_n$, нужно вырезать малые диски вокруг точек worldsheet'а и вычислять инварианты Громова---Виттена, являющиеся подсчётом композиций кобордизмов комплексно одномерных многообразий\que{я правильно понимаю, что основной объект $I(\Sigma, V_1, ..., V_n)$ и соответствующий ему инвариант Громова---Виттена --- это одно и то же?}.

\begin{center}
\begin{tikzpicture}

%\fill[cyan!10] (-0.1,0.4) arc (-45:45:1cm and 1.55cm) arc (-95:40:0.166cm and 0.5cm) arc (-175:-5:2.0cm and 0.53cm) arc (140:275:0.166cm and 0.5cm) arc (135:225:1cm and 1.55cm) arc (85:220:0.166cm and 0.5cm) arc (5:175:2.0cm and 0.53cm) arc (-40:95:0.166cm and 0.5cm);

\fill[cyan!10] (4,0) rectangle (0,3);

\fill[cyan!10] (0.11,3.4) arc (-175:-5:1.9cm and 0.6cm) -- (3.89,2.45) -- (0.11,2.45) -- cycle;
\fill[cyan!10] (3.89,-0.4) arc (5:175:1.9cm and 0.6cm) -- (0.11,0.55) -- (3.89,0.55) -- cycle;

%	\draw[dashed,color=gray] (0,0) arc (-90:90:0.5 and 1.5);% right half of the left ellipse
%	\draw[semithick] (0,0) -- (4,1);% bottom line
%	\draw[semithick] (0,3) -- (4,2);% top line
%	\draw[semithick] (0,0) arc (270:90:0.5 and 1.5);% left half of the left ellipse
	\draw[thick,fill=white] (4,3) ellipse (0.166 and 0.5);% right top ellipse
	\draw[thick,fill=white] (0,3) ellipse (0.166 and 0.5);% left top ellipse
	\draw[thick,fill=white] (4,0) ellipse (0.166 and 0.5);% right bottom ellipse
	\draw[thick,fill=white] (0,0) ellipse (0.166 and 0.5);% left bottom ellipse
\draw[thick,fill=white] (-0.1,0.4) arc (-45:45:1cm and 1.55cm); % left arc
\draw[thick,fill=white] (3.89,-0.4) arc (5:175:1.9cm and 0.6cm); % bottom arc
\draw[thick,fill=white] (4.1,2.6) arc (135:225:1cm and 1.55cm); % right arc
\draw[thick,fill=white] (0.11,3.4) arc (-175:-5:1.9cm and 0.6cm); % top arc

  \draw (-0.5,0) node {\footnotesize $...$};
  \draw (-0.5,3) node {\footnotesize $V_1$};
  \draw (4.5,0) node {\footnotesize $V_n$};
  \draw (4.5,3) node {\footnotesize $V_2$};

  \draw (0.75,1.5) node {\footnotesize $\Sigma_1$};
    \draw (2.75,1.5) node {\footnotesize $\Sigma_2$};

\draw[dashed,thick,color=gray] (1.5,1.5) ellipse (0.1cm and 1.37cm);


%\filldraw[fill=cyan, draw=blue] (6,6) -- (12mm,0mm) arc (0:30:12mm) -- (6,6);
\end{tikzpicture}
\end{center}



Т.е. вот где основной объект принимает значения: $$I (\Sigma, \mu) \in V_1 \otimes ... \otimes V_n \otimes \Big( \mu(\Sigma)/\text{diff} \Big)$$

и при этом есть единственная аксиома: что если эту поверхность разрезать, то
$$I(\Sigma, \mu) = I(\Sigma_1, \mu) \circ I(\Sigma_2, \mu).$$

Тензор энергии-импульса же можно вытащить из такой теории вариацией основного объекта по дифференциалу Бельтрами:

$$T = \frac{\delta I(\Sigma, \mu)}{\delta \mu}$$



Польчинский пишет действие, которое мы назовём действием старой струнной геометрии:

$$S = \int g_{\mu\nu} (x) dx^{\mu} \ast dx^{\nu} + B_{\mu\nu}(x) dx^{\mu} \wedge dx^{\nu},$$

второй член --- поле Калба---Рамона, 2-форма на таргете. Действие новой же струнной геометрии таково:


\be  S = \int\limits_{\Sigma} \Bigg( P_i \bar \partial X^i + \bar P_{\bar i} \partial \bar X^{\bar i} + \tikz[baseline]{\draw[dashed,thick,color=gray] (-4.3,-0.25) rectangle (4.3,0.55);
            \node[anchor=base] (t1)
            {$g^{i \bar j} (x, \bar x) P_i P_{\bar j} +\mu^i_{\bar j} P_i \bar \partial \bar X^{\bar j} + \mu_j^{\bar i} P_{\bar i} \partial X^j +  b_{i \bar j} \partial X^i \bar\partial \bar X^{\bar j}$};}  \Bigg). \ee

С помощью комплексной структуры $J$ можно переходить от старой геометрии к новой и, почти всегда, наоборот:

\be (G, B) \xleftrightarrow{J} (\tikz[baseline]{
            \node[fill=purple!20,anchor=base] (t1)
            {$g$};}, \tikz[baseline]{
            \node[fill=orange!20,anchor=base] (t1)
            {$\mu$};}, \tikz[baseline]{
            \node[fill=green!20,anchor=base] (t1)
            {$\bar \mu$};}, \tikz[baseline]{
            \node[fill=cyan!20,anchor=base] (t1)
            {$b$};}).
            \ee


Поговорим про теорию $$S = \int\limits_{\Sigma} P_i \bar \partial X^i.$$ В ней есть поля размерности $(\bullet, 0)$. Это, конечно, функции $f(x) \in (0,0)$, но также и $(\operatorname{Vect} \oplus \Omega) \in (1,0)$ --- алгебра векторных полей, расширенная своими представлениями. Такие элементы образуют алгебру Ли, назовём её $L$.

Новая геометрия лежит в $L \otimes \bar L$. Именно, произведём следующее несложное вычисление:

$$(V \oplus \Omega^1) \otimes (\bar V \oplus \bar \Omega^1) = $$

\begin{equation*}
= \tikz[baseline]{
            \node[fill=purple!20,anchor=base] (t1)
            {$V \otimes \bar V$};}  \oplus \tikz[baseline]{
            \node[fill=orange!20,anchor=base] (t1)
            {$V \otimes \bar \Omega^1$};} \oplus \tikz[baseline]{
            \node[fill=green!20,anchor=base] (t1)
            {$\Omega^1 \otimes \bar V$};} \oplus \tikz[baseline]{
            \node[fill=cyan!20,anchor=base] (t1)
            {$\Omega^1 \otimes \bar \Omega^1$};}.
\end{equation*}

Как уже стало понятно из боевой раскраски, $g \in V \otimes \bar V$, $\mu \in V \otimes \bar \Omega^1$, $\bar \mu \in \Omega^1 \otimes \bar V$, $b \in \Omega^1 \otimes \bar \Omega^1$.

Новая геометрия <<чуть>> больше старой:

\begin{center}
\begin{tikzpicture}


	\filldraw[fill=blue!70!white, thick] (0,1) ellipse (1cm and 0.2cm); % top ellipse
	\filldraw[fill=orange!70!white, thick] (0,0) ellipse (1.4cm and 0.28cm); % bottom ellipse

\draw[dashed,thick,color=gray] (0,0) ellipse (1cm and 0.2cm); % bottom ellipse

\draw[dashed,thick,color=gray] (1.0,1) -- (1.0,0);
\draw[dashed,thick,color=gray] (-1.0,1) -- (-1.0,0);

\begin{scope}[decoration={amplitude=.4mm,
        segment length=2mm,post length=1mm}]

      \draw[decorate,orange,thick] (1.45,0) -- ++(5:3);
      \draw[decorate,blue,thick] (1.05,1) -- ++(5:4);
    \end{scope}

      \draw (1.45,0) ++(5:3) node[right] {\footnotesize new geometry};
  \draw (1.05,1) ++(5:4) node[right] {\footnotesize old geometry};

\end{tikzpicture}
\end{center}

Например, в старой геометрии годилась только Риччи-плоская метрика.

Новая геометрия существует на 0-мерных схемах. (А 0-мерные схемы --- это по разным причинам хорошо.)



\footnotesize{}

Говоря <<схема>>, мы, в первую очередь, держим в уме следующий пример:

$$\mathbb{C}[x_1, ..., x_n] / I_{F(x)}.$$

Посредством резольвенты Кошуля это эквивалентно $\mathbb{C}[x_1, ..., x_n, \theta]$ с дифференциалом $Q = F \frac{\partial}{\partial \theta}.$


\normalsize{}


Пространство является гомологическим многообразием с гомологическим векторным полем. (Гомологическое векторное поле --- такое векторное поле $Q = v^i (x) \frac{\partial}{\partial x^i}$, что $Q^2 = 0$.) Для вещественно двумерного многообразия с комплексной структурой условие гомологичности векторного поля означает интегрируемость структуры. Деформация многообразия --- это деформация гомологического векторного поля.\que{Тут где-то ещё мимо проходили обобщённые деформации комплексных структур по Баранникову---Концевичу, но, где конкретно, я не понимаю.}

Традиционного пространства в CFT нет.

Пусть есть семейство CFT$_t$. Пусть, когда $t \to t_0$, некоторая группа полей неожиданно приобретает размерность 0.

Назовём эти поля $\tilde \ph$. У них есть операторное разложение

$$\tilde \ph_a (t) \tilde \ph_b (0) = c_{ab}^c (z,t) \tilde \ph_c + ...$$

Мы видим, что пространство возникает алгебраически. Возникает как аффинная схема (<<по Гротендику>>), а не как набор дисков, склеенных между собой.


Классическая физика и связанная с ней дифференциальная геометрия умерли. Фейнман как великий контрреволюционер.

Пусть $\gamma \in L \otimes \bar L, a \in L, \bar a \in \bar L$. Определим скобку $[[,]]$: $$[[a \otimes \bar a, b \otimes \bar b]] := [a, b]_L \otimes [\bar a, \bar b]_{\bar L}$$

Уравнение струнной гравитации, предположительно, выглядит так:

\be (d+Q) (\gamma) + [[\gamma, \gamma]] + \mathcal{O}(\gamma^3) = 0 \ee

$\gamma \in (g, \mu, \bar \mu, b)$, так что это действительно уравнение струнной гравитации (струнной --- потому что с полем Калба---Рамона $b$, гравитации --- потому что с метрикой). У этого уравнения есть решения на схемах, которые можно изучать.

То есть, по-видимому, уравнения струнной гравитации имеют вид уравнения Маурера---Картана.

Не так же ли выглядят уравнения М-теории? Этот вопрос, естественно, открыт.

Изучение этого уравнения и его симметрий --- это и есть более-менее изучение струнной геометрии пространства-времени.


\bigskip
\bigskip


 \end{document}
